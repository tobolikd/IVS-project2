%class
\documentclass[a4paper, 11pt]{article}

%encoding
\usepackage[T1]{fontenc}                %font encoding
\usepackage[utf8]{inputenc}             %script encoding

%packages
\usepackage[czech]{babel}               %language
\usepackage[a4paper, text={17cm,24cm}, left=2cm, top=3 cm]{geometry}		                      %layout
\usepackage{times}                      %font
\usepackage[unicode, hidelinks]{hyperref}	%links
\usepackage{tabularx}
\usepackage{multicol}
\usepackage{multirow}
\usepackage{graphicx}
\usepackage{float}
\usepackage{csquotes}

\newcommand{\pref}[1]{[\ref{#1}]}

\begin{document}

    \begin{center}
        \LARGE Uživatelská příručka ke kalkulačce (projekt IVS)\\[0.4em]

        \Large Tým xtobol06 xtesar43 xsteti05 xhalen00\\[0.4em]

        \Large \today
    \end{center}

    \tableofcontents
    \newpage

    \section{Instalace}
    Podrobný návod, jak nainstalovat kalkulačku prostřednictvím příkazové řádky na debian based distribucích pomocí příkazu apt install:
    \begin{enumerate}
        \item Stáhněte si instalační soubor kalkulačky ve formátu .deb.    
        \item Otevřete terminál a přejděte do složky, kde se instalační soubor nachází.
        \item Zadejte následující příkaz pro instalaci kalkulačky: 
         \textbf{sudo apt install ./calculator.deb}
        \item Stiskněte klávesu Enter, aby se příkaz spustil. Budete vyzváni k zadání hesla.
        \item Zadejte heslo a stiskněte klávesu Enter.
        \item Příkaz apt pak začne instalovat kalkulačku a všechny její závislosti. Počkejte na dokončení procesu.
        \item Po dokončení instalace můžete spustit kalkulačku zadáním příkazu \textbf{calculator} do terminálu nebo vyhledáním v nabídce aplikací pod stejným jménem.
    \end{enumerate}

	\section{Odinstalace}
    Zde je postupný návod, jak odinstalovat vlastní kalkulačku prostřednictvím příkazové řádky na Debian založených distribucích Linuxu pomocí příkazu apt remove:
    Otevřete terminál a zadejte následující příkaz(y) pro odinstalaci kalkulačky:
    \begin{enumerate}
        \item \textbf{sudo apt remove calculator}
        \item Stiskněte klávesu Enter, aby se spustil příkaz. Budete vyzváni k zadání hesla.
        \item Zadejte heslo a stiskněte klávesu Enter.
        \item Příkaz apt pak začne odinstalovat kalkulačku a všechny její závislosti. Počkejte na dokončení procesu.
        \item Po dokončení odinstalace můžete také odstranit konfigurační soubory kalkulačky. Zadejte následující příkaz:
        \item \textbf{sudo apt purge kalkulacka}
        \item Stiskněte klávesu Enter, aby se spustil příkaz. Můžete být vyzváni k zadání hesla.
        \item Zadejte heslo a stiskněte klávesu Enter.
    \end{enumerate}
    \pagebreak
	\section{Použití}
        Tato část popisuje funkce kalkulačky a její použití. Kalkulačka nabízí základní matematické funkce - sčítání, odčítání, násobení, dělení, rovněž také mocninu a n-tou odmocninu. Dále disponuje funkcí pro výpočet kombinačního čísla. Kalkulačku je možné ovládat jak vstupem z klávesnice,tak pomocí myši. Matematický výraz je z kalkulačky možno kdykoliv zkopírovat do schránky pro další použítí. Pokud se v zadaném výrazu vyskytnou chyby, kalkulačka zobrazí příslušné chybové hlášení. 
    \subsection*{Funkce kalkulačky}
    \begin{itemize}
        \item Sčítání (+): Pro sčítání dvou čísel jednoduše zadejte první číslo, pak stiskněte tlačítko "+", poté zadejte druhé číslo a stiskněte "=" pro získání výsledku.
        \item Odčítání (-): Pro odečtení dvou čísel zadejte první číslo, stiskněte tlačítko "-", poté zadejte druhé číslo a stiskněte "=".
        \item Násobení (*): Pro násobení dvou čísel zadejte první číslo, stiskněte tlačítko "*", poté zadejte druhé číslo a stiskněte "=".
        \item Dělení (/): Pro dělení dvou čísel zadejte první číslo, stiskněte tlačítko "/", zadejte druhé číslo a stiskněte "=".
        \item N-tá odmocnina (root): Pro výpočet n-té odmocniny čísla zadejte nejdříve řád odmovniny, stiskněte tlačítko se symbolem odmocniny a následně zadejte výraz pod odmocninou. Delší výraz je nutné obalit do závorek. V případě druhé odmocniny není potřeba provádět první krok a lze zadat pouze výraz pod odmocninou.
        \item Umocnění (\verb|^|): Pro umocnění čísla zadejte základ umocnění, stiskněte tlačítko "\verb|^|", zadejte exponent a stiskněte "=".
        \item Kombinační číslo (nCr): Pro výpočet kombinačního čísla k z n prvků zadejte nejdříve hodnotu k, stiskněte tlačítko "nCr" a pak zadejte druhou hodnotu n.
        \item Závorky ( ): Závorky používejte pro obalení výrazů s přednostním výpočtem. Kalkulačka vyhodnocuje výraz standardně zleva do prava a používá běžnou prioritu operací. Neuzavřené závorky na konci výrazu kalkulačka automaticky doplní.
        \item Příklad: Pro výpočet výrazu 3+4x5 kalkulačka spočítá výraz s ohledem na prioritu matematických funkcí jako 3+(4x5). Pro výpočet začínající s operací ščítání je tedy v tomto případě nutné zaobalit první část výrazu do závorek: (3+4)*5, aby kalkulačka upřednostnila operaci sčítání před násobením.
    \end{itemize}
    \subsection*{Chybové hlášení}
     Pokud je výraz do kalkulačky zadán nesprávně, může být místo výsledku zobrazeno chybové hlášení, jako například 'NUMBER OUT OF RANGE' nebo 'INVALID EXPRESSION', což indikuje, že je nutné upravit výraz pro správný výpočet. 
    \subsection*{Zadávání hodnot}
    Kromě vstupu z myši, můžete zadávat vstup do kalkulačky přímo z klávesnice, což vám umožní rychlejší a efektivnější práci s kalkulačkou. 
    \subsection*{Výstup kalkulačky}
    Chcete-li výraz z kalkulačky zkopírovat do schránky, jednoduše označte požadovanou část výrazu z textového pole a poté stiskněte klávesy \textbf{ctrl + c}.
\end{document}

