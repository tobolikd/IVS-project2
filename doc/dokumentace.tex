%class
\documentclass[a4paper, 11pt]{article}

%encoding
\usepackage[T1]{fontenc}           %font encoding
\usepackage[utf8]{inputenc}         %script encoding

%packages
\usepackage[czech]{babel}           %language
\usepackage[a4paper, text={17cm,24cm}, left=2cm, top=3 cm]{geometry}		%layout
\usepackage{times}					%font
\usepackage[unicode, hidelinks]{hyperref}	%links
\usepackage{tabularx}
\usepackage{multicol}
\usepackage{multirow}
\usepackage{graphicx}
\usepackage{float}
\usepackage{csquotes}

\newcommand{\pref}[1]{[\ref{#1}]}

\begin{document}

    \begin{center}
        \LARGE Uživatelská příručka ke kalkulačce do projektu IVS\\[0.4em]

        \Large Tým xtobol06 xtesar43 xsteti05 xhalen00\\[0.4em]

        \Large \today
    \end{center}

    \tableofcontents
    \newpage

    \section{Instalace}
    Kalkulačku je možné nainstalovat příkazem \em{sudo apt install calculator.deb}
	\section{Odinstalace}
    Kalkulačku je možné odinstalovat příkazem \em{sudo apt remove calculator.deb}
	\section{Manuální odinstalace}
    Kalkulačku je možné odinstalovat příkazem \em{sudo apt remove calculator.deb}
	\section{Použití}
        Tento manuál popisuje základní funkce kalkulačky a její použití. Kalkulačka nabízí základní matematické funkce - sčítání, odčítání, násobení, dělení, rovněž také mocninu a n-tou odmocninu. Dále disponuje funkcí pro výpočet kombinačního čísla. Kalkulačku je možné ovládat jak vstupem z klávesnice,tak pomocí myši. Matematický výraz je z kalkulačky možno kdykoliv zkopírovat do schránky pro další použítí mimo kalkulačku. Pokud se v zadaném výrazu vyskytnou chyby, kalkulačka zobrazí příslušné chybové hlášení. 
        Základní použítí kalkulačky (podrobný popis):
    \begin{itemize}
        \item Sčítání (+): Pro sčítání dvou čísel jednoduše zadejte první číslo, pak stiskněte tlačítko "+", poté zadejte druhé číslo a stiskněte "=" pro získání výsledku.
        \item Odčítání (-): Pro odečtení dvou čísel zadejte první číslo, stiskněte tlačítko "-", poté zadejte druhé číslo a stiskněte "=".
        \item Násobení (*): Pro násobení dvou čísel zadejte první číslo, stiskněte tlačítko "*", poté zadejte druhé číslo a stiskněte "=".
        \item Dělení (/): Pro dělení dvou čísel zadejte první číslo, stiskněte tlačítko "/", zadejte druhé číslo a stiskněte "=".
        \item N-tá odmocnina (root): Pro výpočet n-té odmocniny čísla zadejte nejdříve řád odmovniny, stiskněte tlačítko se symbolem odmocniny a následně zadejte výraz pod odmocninou. Delší výraz je nutné obalit do závorek. V případě druhé odmocniny není potřeba provádět první krok a lze zadat pouze výraz pod odmocninou.
        \item Umocnění (\verb|^|): Pro umocnění čísla zadejte základ umocnění, stiskněte tlačítko "\verb|^|", zadejte exponent a stiskněte "=".
        \item Kombinační číslo (nCr): Pro výpočet kombinačního čísla k z n prvků zadejte nejdříve hodnotu k, stiskněte tlačítko "nCr" a pak zadejte druhou hodnotu n.
        \item Závorky ( ): Závorky používejte pro obalení výrazů s přednostním výpočtem. Kalkulačka vyhodnocuje výraz standardně zleva do prava a používá běžnou prioritu operací. Neuzavřené závorky na konci výrazu kalkulačka automaticky doplní.
        \item Příklad: Pro výpočet výrazu 3+4x5 kalkulačka spočítá výraz s ohledem na prioritu matematických funkcí jako 3+(4x5). Pro výpočet začínající s operací ščítání je tedy v tomto případě nutné zaobalit první část výrazu do závorek: (3+4)*5, aby kalkulačka upřednostnila operaci sčítání před násobením.
    \end{itemize}
    Tlačítko 'ANS' ukládá poslední výsledek, což usnadňuje opakované použití v dalších výpočtech. Pokud je výraz zadán nesprávně, může být místo výsledku zobrazeno chybové hlášení, jako například 'ZERO DIVISION' nebo 'INVALID SYNTAX', což indikuje, že je nutné upravit výraz pro správný výpočet. Kromě vstupu z myši, můžete zadávat vstup do kalkulačky přímo z klávesnice, což vám umožní rychlejší a efektivnější práci s kalkulačkou. Chcete-li výraz z kalkulačky zkopírovat do schránky, jednoduše označte požadovanou část výrazu z textového pole a poté stiskněte klávesy ctrl + c.
\end{document}

